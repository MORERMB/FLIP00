%%
%% This is file `tikzposter-template.tex',
%% generated with the docstrip utility.
%%
%% The original source files were:
%%
%% tikzposter.dtx  (with options: `tikzposter-template.tex')
%%
%% This is a generated file.
%%
%% Copyright (C) 2014 by Pascal Richter, Elena Botoeva, Richard Barnard, and Dirk Surmann
%%
%% This file may be distributed and/or modified under the
%% conditions of the LaTeX Project Public License, either
%% version 2.0 of this license or (at your option) any later
%% version. The latest version of this license is in:
%%
%% http://www.latex-project.org/lppl.txt
%%
%% and version 2.0 or later is part of all distributions of
%% LaTeX version 2013/12/01 or later.
%%


\documentclass{tikzposter} %Options for format can be included here

\usepackage{todonotes}

\usepackage[tikz]{bclogo}
\usepackage{lipsum}
\usepackage{amsmath}

\usepackage{booktabs}
\usepackage{longtable}
\usepackage[absolute]{textpos}
\usepackage[it]{subfigure}
\usepackage{graphicx}
\usepackage{cmbright}
%\usepackage[default]{cantarell}
%\usepackage{avant}
%\usepackage[math]{iwona}
\usepackage[math]{kurier}
\usepackage[T1]{fontenc}


%% add your packages here
\usepackage{hyperref}
% for random text
\usepackage{lipsum}
\usepackage[english]{babel}
\usepackage[pangram]{blindtext}

\colorlet{backgroundcolor}{blue!10}

 % Title, Author, Institute
\title{April Tabular Playground Series - Your Baseline Model}
\author{Jinzhu Liu$^1$}
\institute{$^1$ Nanjing University of Science and Technology, China \\
}
%\titlegraphic{logos/tulip-logo.eps}

%Choose Layout
\usetheme{Wave}

%\definebackgroundstyle{samplebackgroundstyle}{
%\draw[inner sep=0pt, line width=0pt, color=red, fill=backgroundcolor!30!black]
%(bottomleft) rectangle (topright);
%}
%
%\colorlet{backgroundcolor}{blue!10}

\begin{document}


\colorlet{blocktitlebgcolor}{blue!23}

 % Title block with title, author, logo, etc.
\maketitle

\begin{columns}
 % FIRST column
\column{0.5}% Width set relative to text width

%%%%%%%%%% -------------------------------------------------------------------- %%%%%%%%%%
 %\block{Main Objectives}{
%  	      	\begin{enumerate}
%  	      	\item Formalise research problem by extending \emph{outlying aspects mining}
%  	      	\item Proposed \emph{GOAM} algorithm is to solve research problem
%  	      	\item Utilise pruning strategies to reduce time complexity
%  	      	\end{enumerate}
%%  	      \end{minipage}
%}
%%%%%%%%%% -------------------------------------------------------------------- %%%%%%%%%%


%%%%%%%%%% -------------------------------------------------------------------- %%%%%%%%%%
\block{Introduction}{
 	Predict Whether or not A Passenger Survived the Inking of the Synthanic.
  	
  	\begin{description}
  	\item[Evaluation Goal] We task is to predict whether or not a passenger survived the sinking of the Synthanic (a synthetic, much larger dataset based on the actual Titanic dataset). For each row in the test set, you must predict a 0 or 1 value for the target.We score is the percentage of passengers you correctly predict. This is known as accuracy.
  	
  	\item[Outlier Detection] aims to identify all possible outliers in the dataset,
    without explaining why or how they are different.
  	\end{description}

  	In this paper,
 	It is important to aim to start somewhere and identify it as the first criterion to compare with existing progress. This helps you create a \emph{baseline model} and get a  baseline score.
}
%%%%%%%%%% -------------------------------------------------------------------- %%%%%%%%%%


%%%%%%%%%% -------------------------------------------------------------------- %%%%%%%%%%
\block{Variable Notes}{
\begin{itemize}
    \item
	 \emph{pclass}: A proxy for socio-economic status (SES) 1st = Upper 2nd = Middle 3rd = Lower \\
	\emph{age}: Age is fractional if less than 1. If the age is estimated, is it in the form of xx.5 \\
	\emph{sibsp}: The dataset defines family relations in this way.Sibling = brother, sister, stepbrother, stepsister Spouse = husband, wife (mistresses and fiancés were ignored) \\
	\emph{parch}: The dataset defines family relations in this way.Parent = mother, father Child = daughter, son, stepdaughter, stepson Some children travelled only with a nanny, therefore parch=0 for them.

    \item
    \emph{Loading the Data},
\end{itemize}
\vspace{.5cm}
\begin{center}
	\begin{tabular}{c| c c c c c c c c}
		\toprule
		%\centering
		{} & \texttt{Passengerid}  & \texttt{Survived} & \texttt{Pclass} & \texttt{Name}  & \texttt{Sex} & \texttt{Age}  & \texttt{SibSp}  & \texttt{Parch} \\
		\midrule
		$0$
		&  {$0$} &  {$1$} &  {$1$} &  {Oconnor, Frankie} &  {male} &  {NaN} &  {$2$} &  {$0$} \\
		$1$
		&  {$1$} &  {$0$}&  {$3$}&  {Bryan, Drew} &  {male} &  {NaN} &  {$0$} &  {$0$} \\
		$2$
		&  {$2$} &  {$0$} &  {$3$} &  {Owens, Kenneth} &  {male} &  {$0.33$} &  {$1$} &  {$2$} \\
		$3$
		&  {$3$} &  {$0$}&  {$3$}&  {Kramer, James} &  {male} &  {$19.00$} &  {$0$} &  {$0$} \\
		$4$
		&  {$4$} &  {$1$} &  {$3$} &  {Bond, Michael} &  {male} &  {$25.00$} &  {$0$} &  {$0$} \\
		\bottomrule
	\end{tabular}
\end{center}
}
%%%%%%%%%% -------------------------------------------------------------------- %%%%%%%%%%


%%%%%%%%%% -------------------------------------------------------------------- %%%%%%%%%%

%\note{Note with default behavior}

%\note[targetoffsetx=12cm, targetoffsety=-1cm, angle=20, rotate=25]
%{Note \\ offset and rotated}

 % First column - second block


%%%%%%%%%% -------------------------------------------------------------------- %%%%%%%%%%
\block{Data analysis and processing}{
	Inquire NaN Values.

  	
\begin{center}
\fbox{ 
\parbox{0.43\textwidth}{ 
\begin{center}
\begin{itemize}
\item
list out the columns holding Numerical Values - \textcolor{orange}{1.Age  2.Fare}
\item
The remaining columns do not hold numerical values. Let's explore the distribution of the numerical values a bit before we replace their NaN values
\end{itemize}
\end{center}
} 
}
\end{center}
		
\begin{description}
  	\item[Fare Chart information]
	At first, we planned to distinguish classes according to fare,Seems like the ticket to the titanic did not have any fixed price for any class in particular.so it is intended to scale the data and then use the average to impute.
\end{description}

\begin{description}
\item[Age Chart information]
	It can be seen from the figure that the missing value can be filled by the median.
\end{description}
\vspace{.5cm}
\begin{minipage}{\textwidth}
	
	\begin{minipage}[t]{0.22\textwidth}
		\makeatletter\def\@captype{table}
	\begin{center}	
	\begin{tabular}{c|c}
	\toprule
	%\centering
	\midrule
	{Count}
	&  {$99866.000000$} \\
	{mean}
	&  {$43.92933$} \\
	{std}
	&  {$69.58882$} \\
	{min}
	&  {$0.68000$} \\
	{$25\%$}
	&  {$10.04000$} \\
	{$50\%$}
	&  {$24.46000$} \\
	{$75\%$}
	&  {$33.50000$} \\
	{Max}
	&  {$744.66000$} \\
	\bottomrule
\end{tabular}
\end{center}	
	\end{minipage}
	\begin{minipage}[t]{0.22\textwidth}
		\makeatletter\def\@captype{table}
	\begin{center}	
	\begin{tabular}{c|c}
	\toprule
	%\centering
	\midrule
	{Count}
	&  {$96708.000000$} \\
	{mean}
	&  {$38.355472$} \\
	{std}
	&  {$18.313556$} \\
	{min}
	&  {$ 0.080000$} \\
	{$25\%$}
	&  {$25.000000$} \\
	{$50\%$}
	&  {$39.000000$} \\
	{$75\%$}
	&  {$ 53.000000$} \\
	{Max}
	&  {$87.000000$} \\
	\bottomrule
\end{tabular}
\end{center}	
\end{minipage}
\end{minipage}


}

%%%%%%%%%% -------------------------------------------------------------------- %%%%%%%%%%


% SECOND column
\column{0.5}
 %Second column with first block's top edge aligned with with previous column's top.

%%%%%%%%%% -------------------------------------------------------------------- %%%%%%%%%%
\block{Data analysis and processing}{
\begin{description}
    \item
    Second,
    We will use \emph{KNN} to perform missing interpolation for Embarked.
\end{description}
\vspace{.5cm}
\begin{center}	
	\begin{tabular}{c|c}
		\toprule
		%\centering
		\midrule
		{PassengerId}
		&  {$0$} \\
		{Survived}
		&  {$0$} \\
		{Pclass}
		&  {$0$} \\
		{Name}
		&  {$0$} \\
		{Sex}
		&  {$0$} \\
		{Age}
		&  {$0$} \\
		{SibSp}
		&  {$0$} \\
		{Parch}
		&  {$0$} \\
		{Ticket}
		&  {$4623$} \\
		{Fare}
		&  {$0$} \\
		{Cabin}
		&  {$67866$} \\
		{Embarked}
		&  {$0$} \\
		\bottomrule
	\end{tabular}
\end{center}
\vspace{.5cm}
\begin{description}
  	\item[Imputing Values]
    In this step,
	from this we can see that variables SibSp, Parch , PassengerId have a very small
	correlation coefficient as compared to others. We will be dropping these variables.
\end{description}
}
%%%%%%%%%% -------------------------------------------------------------------- %%%%%%%%%%
% Second column - first block


%%%%%%%%%% -------------------------------------------------------------------- %%%%%%%%%%
\block[titleleft]{Experiment}
{
\begin{description}
  	\item[Titanic Dataset] contains 100000 tourists and corresponding 12 attributes.
\end{description}
\vspace{.5cm}
\begin{center}
	\fbox{ 
	\parbox{0.43\textwidth}{ 
	\begin{center}
	\begin{itemize}
	\item
	First, we set up several classifier models to make a prediction. Then we use the
	training set to fit the model I built. After fitting, I use the fitted model to predict the remaining data in the training set and calculate its accuracy, weight, etc. Then fuse multiple groups of models, stack the fused model with the logistic regression model,and then fit the training set to get the prediction score. Use this model to predict our test set and see the prediction results of our test set.
	\end{itemize}
	\end{center}
} 
}
\end{center}
\vspace{.5cm}
\begin{description}
    \item
    we try Stacking Classifier and Logistic Regression on the test data.
\end{description}
\vspace{.5cm}
\begin{center}
	\begin{tabular}{c| c c c c}
		\toprule
		%\centering
		{}  & \texttt{precision} & \texttt{recall}  & \texttt{f1-score} & \texttt{support} \\
		\midrule
		$0$
		&  {$0.79$} &  {$0.80$} &  {$0.79$} &  {$11336$} \\
		$1$
		&  {$0.73$} &  {$0.72$} &  {$0.72$} &  {$8664$} \\
		accuracy
		&  {} &  {} &  {$0.76$} &  {$20000$} \\
		macro avg
		&  {$0.76$} &  {$0.76$} &  {$0.76$} &  {$20000$} \\
		weight avg
		&  {$0.76$} &  {$0.76$} &  {$0.76$} &  {$20000$} \\
		\bottomrule
	\end{tabular}
\end{center}
\vspace{.5cm}
\begin{description}
	\item
	First, let’s see if there are missing values in the test, and fill in the missing values
	in the test in the same way as train. Finally, the well-fitting model is used to make
	predictions.
\end{description}

}
%%%%%%%%%% -------------------------------------------------------------------- %%%%%%%%%%


% Second column - second block
%%%%%%%%%% -------------------------------------------------------------------- %%%%%%%%%%
\block[titlewidthscale=1, bodywidthscale=1]
{Conclusion}
{
\begin{description}
  \item[Prediction effect]
  Finally, the highest accuracy in LogisticRegression Model is 0.76.

  \item[Project experience]
	A relatively basic Kaggle project was selected, the purpose is to be familiar with the
	Kaggle project, deeply analyze and understand each line of the project process, this
	project has done more processing on the step of data feature processing, and learned
	a lot from it.

  \item[Direction to be improved]
  The work can also be further refined to improve the accuracy of prediction, for
  example, in the process of processing the age column, the age can be segmented
  according to the size of the age, and it is felt that the size of the age has a certain
  relationship with the size of the final survival rate.
\end{description}
}
%%%%%%%%%% -------------------------------------------------------------------- %%%%%%%%%%

\end{columns}


%%%%%%%%%% -------------------------------------------------------------------- %%%%%%%%%%
%[titleleft, titleoffsetx=2em, titleoffsety=1em, bodyoffsetx=2em,%
%roundedcorners=10, linewidth=0mm, titlewidthscale=0.7,%
%bodywidthscale=0.9, titlecenter]

%\colorlet{noteframecolor}{blue!20}
\colorlet{notebgcolor}{blue!20}
\colorlet{notefrcolor}{blue!20}
\note[targetoffsetx=-13cm, targetoffsety=-12cm,rotate=0,angle=180,radius=8cm,width=.96\textwidth,innersep=.4cm]
{
\begin{minipage}{0.3\linewidth}
\centering
\includegraphics[width=24cm]{./graphics/logos/tulip-wordmark.eps}
\end{minipage}
\begin{minipage}{0.7\linewidth}
{ \centering
 April Tabular Playground Series - Your Baseline Model
}
\end{minipage}
}
%%%%%%%%%% -------------------------------------------------------------------- %%%%%%%%%%


\end{document}

%\endinput
%%
%% End of file `tikzposter-template.tex'.
