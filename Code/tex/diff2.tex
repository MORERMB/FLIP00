%=================================================================
\section{Introduction}\label{sec-intro}


%\todo{Narrow down to a topic; Dig a hole; Fill the hole}
\todo{Formula for Introduction}



%\gangli{``narrow in on topic'' reminds you 
%that readers and reviewers only know that this is a AI or HTM research paper (and maybe have read the title/abstract). 
%You need to help them figure out what topic and area of research paper this is. 
%You _don't_ need to wax poetic about the topic's importance.}

%\gangli{`dig a hole'' reminds you that 
%you need to convince the reader that there's a problem with the state of the world. 
%Prior work may exist but it's either missing something important or there's a missing opportunity. 
%The reader should be drooling for a bright future just out of reach.}

%\gangli{``fill the hole'' reminds you to show the reader 
%how and why the paper they're reading will fix these problems and deliver us into a better place. 
%You don't need a whirlwind summary of the technical details, 
%but you need readers convinced (and in a good mood) to keep reading.}

\gangli{A good paper introduction is fairly formulaic. 
If you follow a simple set of rules, 
you can write a very good introduction. 
The following outline can be varied. 
For example, 
you can use two paragraphs instead of one, 
or you can place more emphasis on one aspect of the intro than another. 
But in all cases, 
all of the points below need to be covered in an introduction, 
and in most papers, 
you don't need to cover anything more in an introduction.}



%\todo{The importance of the area}
%\blindtext
\todo{Motivation}
\DIFdelbegin \DIFdel{At a high level, what is the problem area you are working in and why is it important? 
It is important to setthe larger context here.Why is the problem of interest and importance to the larger community?
}\DIFdelend \DIFaddbegin \DIFadd{Predict Whether or not A Passenger Survived the Inking of the Synthanic.
We task is to predict whether or not a passenger survived the sinking of the Synthanic (a synthetic, much larger dataset based on the actual Titanic dataset). For each row in the test set, you must predict a 0 or 1 value for the target.We score is the percentage of passengers you correctly predict. This is known as accuracy.
}\DIFaddend 


%\todo{The problems faced by most current methods}
%\blindtext
\todo{What is the specific problem considered in this paper?}
This paragraph  \DIFdelbegin \DIFdel{narrows down the topic area of the paper.
In the first paragraph you have established general context and importance. 
Here you establish specific context and background.
}\DIFdelend \DIFaddbegin \DIFadd{aims to identify all possible outliers in the dataset,
without explaining why or how they are different.
}\DIFaddend 

%\todo{What can be addressed by existing methods; Why those problems are challenges to existing methods?}
%\blindtext
\todo{Contribution}
\DIFdelbegin \DIFdel{"In this paper, we show that ...". 
This is the key paragraph in the intro - you summarize, in one paragraph, 
what are the main contributions of your paper given the context 
you have established in paragraphs 1 and 2. 
What is the general approach taken? 
Why are the specific results significant? 
This paragraph must be really good}\DIFdelend \DIFaddbegin \DIFadd{A relatively basic Kaggle project was selected, the purpose is to be familiar with the Kaggle project, deeply analyze and understand each line of the project process, this project has done more processing on the step of data feature processing, and learned a lot from it}\DIFaddend .

\DIFdelbegin \DIFdel{You should think about how to structure these one or 
two paragraph summaries of what your paper is all about.
If there are two or three main results, 
then you might consider itemizing them with bullets or in test. 
}\DIFdelend \DIFaddbegin \DIFadd{There are some variables that need to be introduced.
}\DIFaddend \begin{itemize}
	\item \DIFdelbegin \DIFdel{e.g., First ...
	}\DIFdelend \DIFaddbegin \DIFadd{Pclass: A proxy for socio-economic status (SES) 1st = Upper 2nd = Middle 3rd = Lower
	}\DIFaddend \item \DIFdelbegin \DIFdel{e.g., Second ...
	}\DIFdelend \DIFaddbegin \DIFadd{age: Age is fractional if less than 1. If the age is estimated, is it in the form of xx.5
	}\DIFaddend \item \DIFdelbegin \DIFdel{e.g., Third ...
}\DIFdelend \DIFaddbegin \DIFadd{sibsp: The dataset defines family relations in this way.Sibling = brother, sister, stepbrother, stepsister Spouse = husband, wife (mistresses and fiancés were ignored)
	}\item \DIFadd{parch: The dataset defines family relations in this way.Parent = mother, father Child = daughter, son, stepdaughter, stepson Some children travelled only with a nanny, therefore parch=0 for them
}\DIFaddend \end{itemize}
\DIFdelbegin \DIFdel{If the results fall broadly into two categories, you can bring out that distinction here. 
For example, "Our results are both theoretical and applied in nature. 
(two sentences follow, one each on theory and application)
	"
}\DIFdelend 

\DIFaddbegin \section{\DIFadd{Preliminaries}} \label{sec-preliminaries}

\DIFadd{The data in the dataset can be roughly divided into two types: numerical type and non-numerical type.
list out the columns holding Numerical Values - \textcolor{orange}{1.Age  2.Fare} }\\
\DIFadd{The remaining columns do not hold numerical values. Let's explore the distribution of the numerical values a bit before we replace their NaN values.
}

\DIFaddend %\todo{What provides the motivation of this work? What are the research issues? What is the rationale of this work? }
%\blindtext
\DIFdelbegin %DIFDELCMD < \todo{At a high level what are the differences in what you are doing, and what others have done? }
%DIFDELCMD < %%%
\DIFdel{Keep this at a high level, you can refer to a future section where specific details and differences will be given. 
But it is important for the reader to know at a high level, what is new about this work compared to other work in the area.
}\DIFdelend 

%DIF < \todo{What we have done and what are the contributions.}
%DIF < \blindtext
\DIFdelbegin %DIFDELCMD < \todo{A roadmap for the rest of the paper}
%DIFDELCMD < %%%
\DIFdel{"The remainder of this paper is structured as follows. .." 
Give the reader a roadmap for the rest of the paper.
Avoid redundant phrasing, "In Section 2, In section 3, .
.. In Section 4, ... " etc.
}\DIFdelend \DIFaddbegin \todo{Data analysis and processing}
\DIFadd{First, we first analyze and process numerical data: age and fare.
}\DIFaddend 

\gangli{A few general tips:
Don't spend a lot of time into the introduction 
telling the reader about what you don't do in the paper. 
Be clear about what you do do.
Does each paragraph have a theme sentence that sets the stage for the entire paragraph? Are the sentences and topics in the paragraph all related to each other?}

\gangli{Does each paragraph have a theme 
sentence that sets the stage for the entire paragraph? 
Are the sentences and topics in the paragraph all related to each other?}

\gangli{Do all of your tenses match up in a paragraph?}
\DIFaddbegin \vspace{0.5cm}
\begin{minipage}{\textwidth}
	\begin{minipage}[t]{0.5\textwidth}
		\makeatletter\def\@captype{table}
		\begin{center}	
			\begin{tabular}{c|c}
				\toprule
				%DIF > \centering
				\midrule
				{\DIFadd{Count}}
				&  {\DIFadd{$99866.000000$}} \\
				{\DIFadd{mean}}
				&  {\DIFadd{$43.92933$}} \\
				{\DIFadd{std}}
				&  {\DIFadd{$69.58882$}} \\
				{\DIFadd{min}}
				&  {\DIFadd{$0.68000$}} \\
				{\DIFadd{$25\%$}}
				&  {\DIFadd{$10.04000$}} \\
				{\DIFadd{$50\%$}}
				&  {\DIFadd{$24.46000$}} \\
				{\DIFadd{$75\%$}}
				&  {\DIFadd{$33.50000$}} \\
				{\DIFadd{Max}}
				&  {\DIFadd{$744.66000$}} \\
				\bottomrule
			\end{tabular}
		\end{center}	
	\end{minipage}
	\begin{minipage}[t]{0.5\textwidth}
		\makeatletter\def\@captype{table}
		\begin{center}	
			\begin{tabular}{c|c}
				\toprule
				%DIF > \centering
				\midrule
				{\DIFadd{Count}}
				&  {\DIFadd{$96708.000000$}} \\
				{\DIFadd{mean}}
				&  {\DIFadd{$38.355472$}} \\
				{\DIFadd{std}}
				&  {\DIFadd{$18.313556$}} \\
				{\DIFadd{min}}
				&  {\DIFadd{$ 0.080000$}} \\
				{\DIFadd{$25\%$}}
				&  {\DIFadd{$25.000000$}} \\
				{\DIFadd{$50\%$}}
				&  {\DIFadd{$39.000000$}} \\
				{\DIFadd{$75\%$}}
				&  {\DIFadd{$ 53.000000$}} \\
				{\DIFadd{Max}}
				&  {\DIFadd{$87.000000$}} \\
				\bottomrule
			\end{tabular}
		\end{center}	
	\end{minipage}
\end{minipage}
\vspace{0.5cm}
\DIFaddend 

\DIFdelbegin \DIFdel{Test citation~\mbox{%DIFAUXCMD
\cite{BL12J01}}\hskip0pt%DIFAUXCMD
.}%DIFDELCMD < \begin{JournalOnly}
%DIFDELCMD < %%%
\DIFdel{and ~\mbox{%DIFAUXCMD
\citep{BJL11J01} }\hskip0pt%DIFAUXCMD
or~\mbox{%DIFAUXCMD
\citet{BJL11J01}}\hskip0pt%DIFAUXCMD
.
}%DIFDELCMD < \end{JournalOnly}
%DIFDELCMD < %%%
\DIFdelend \DIFaddbegin \begin{description}
	\item[\DIFadd{Fare Chart information}]
	\DIFadd{At first, we planned to distinguish classes according to fare,Seems like the ticket to the titanic did not have any fixed price for any class in particular.so it is intended to scale the data and then use the average to impute.
}\end{description}
\DIFaddend 

\DIFdelbegin \DIFdel{This is for~}%DIFDELCMD < \cref{tbl:overall-experiments}%%%
\DIFdel{,}%DIFDELCMD < \todo[fancyline]{Testing.}
%DIFDELCMD < %%%
\DIFdel{and this is for ~}%DIFDELCMD < \cref{sec-conclusions}%%%
\DIFdel{.
}%DIFDELCMD < \todo[noline]{A note with no line back to the text.}%%%
%DIF < 
%DIFDELCMD < \gangli{This is comment from Gang.}
%DIFDELCMD < \qwu{Response from QW}
%DIFDELCMD < %%%
\DIFdelend \DIFaddbegin \begin{description}
	\item[\DIFadd{Age Chart information}]
	\DIFadd{It can be seen from the figure that the missing value can be filled by the median.
}\end{description}
\vspace{.5cm}
\DIFaddend 

\DIFdelbegin \DIFdel{Number:
}%DIFDELCMD < \num{123}%%%
\DIFdel{.
}%DIFDELCMD < \numlist{10;30;50;70}%%%
\DIFdel{,
	}%DIFDELCMD < \numrange{10}{30}%%%
\DIFdel{, }%DIFDELCMD < \SIlist{10;30;45}{\metre}%%%
\DIFdel{, and
}%DIFDELCMD < \SI{10}{\percent}
%DIFDELCMD < %%%
\DIFdelend \DIFaddbegin \begin{itemize}
\item \DIFadd{Second,We will use }\emph{\DIFadd{KNN}} \DIFadd{to perform missing interpolation for Embarked.
}\end{itemize}
\vspace{.5cm}
\begin{center}	
	\begin{tabular}{c|c}
		\toprule
		%DIF > \centering
		\midrule
		{\DIFadd{PassengerId}}
		&  {\DIFadd{$0$}} \\
		{\DIFadd{Survived}}
		&  {\DIFadd{$0$}} \\
		{\DIFadd{Pclass}}
		&  {\DIFadd{$0$}} \\
		{\DIFadd{Name}}
		&  {\DIFadd{$0$}} \\
		{\DIFadd{Sex}}
		&  {\DIFadd{$0$}} \\
		{\DIFadd{Age}}
		&  {\DIFadd{$0$}} \\
		{\DIFadd{SibSp}}
		&  {\DIFadd{$0$}} \\
		{\DIFadd{Parch}}
		&  {\DIFadd{$0$}} \\
		{\DIFadd{Ticket}}
		&  {\DIFadd{$4623$}} \\
		{\DIFadd{Fare}}
		&  {\DIFadd{$0$}} \\
		{\DIFadd{Cabin}}
		&  {\DIFadd{$67866$}} \\
		{\DIFadd{Embarked}}
		&  {\DIFadd{$0$}} \\
		\bottomrule
	\end{tabular}
\end{center}
\vspace{.5cm}
\begin{description}
	\item[\DIFadd{Imputing Values}]
	\DIFadd{In this step,
	from this we can see that variables SibSp, Parch , PassengerId have a very small
	correlation coefficient as compared to others. We will be dropping these variables.
}\end{description}
\DIFaddend 

\DIFdelbegin %DIFDELCMD < \missingfigure[figcolor=white]{Testing figcolor}
%DIFDELCMD < 

%DIFDELCMD < \begin{ConferenceOnly}
%DIFDELCMD < %%%
\DIFdel{We have }%DIFDELCMD < \SI{10}{\hertz}%%%
\DIFdel{,
}%DIFDELCMD < \si{\kilogram\metre\per\second}%%%
\DIFdel{,
the range: }%DIFDELCMD < \SIrange{10}{100}{\hertz}%%%
\DIFdel{. $\nicefrac[]{1}{2}$.
}%DIFDELCMD < 

%DIFDELCMD < \missingfigure{Make a sketch of the structure of a trebuchet.}
%DIFDELCMD < 

%DIFDELCMD < \end{ConferenceOnly}
%DIFDELCMD < 

%DIFDELCMD < %%%
\DIFdel{For~}%DIFDELCMD < \cref{eq:test}%%%
\DIFdel{,
as shown below:
}%DIFDELCMD < 

%DIFDELCMD < %%%
\begin{displaymath}\DIFdel{%DIFDELCMD < \label{eq:test}%%%
a = b \times \sqrt{ab}
}\end{displaymath}%DIFAUXCMD
%DIFDELCMD < 

%DIFDELCMD < \blindmathpaper
%DIFDELCMD < 

%DIFDELCMD < %%%
\section{\DIFdel{Preliminaries}} %DIFAUXCMD
\addtocounter{section}{-1}%DIFAUXCMD
%DIFDELCMD < \label{sec-preliminaries}
%DIFDELCMD < 

%DIFDELCMD < \blindtext
%DIFDELCMD < 

%DIFDELCMD < \gliMarker  %%%
%DIF < TODO: GLi Here
%DIFDELCMD < 

%DIFDELCMD < %%%
\section{\DIFdel{Method}} %DIFAUXCMD
\addtocounter{section}{-1}%DIFAUXCMD
%DIFDELCMD < \label{sec-method}
%DIFDELCMD < 

%DIFDELCMD < \blindtext
%DIFDELCMD < \blindlist{itemize}[3]
%DIFDELCMD < \blinditemize
%DIFDELCMD < \blindenumerate
%DIFDELCMD < 

%DIFDELCMD < \blindmathtrue
%DIFDELCMD < \blindmathfalse
%DIFDELCMD < \blinddescription
%DIFDELCMD < 

%DIFDELCMD < \qwuMarker %%%
%DIF < TODO: QWu Here
%DIFDELCMD < 

%DIFDELCMD < %%%
\DIFdelend \section{\DIFdelbegin \DIFdel{Experiment and Analysis}\DIFdelend \DIFaddbegin \DIFadd{ML Models Implementation}\DIFaddend } \label{sec-experiment}
\DIFdelbegin %DIFDELCMD < 

%DIFDELCMD < \begin{table}  \centering
%DIFDELCMD <   %%%
%DIFDELCMD < \caption{%
{%DIFAUXCMD
\DIFdelFL{Precision Comparison on Event Detection Methods}}
  %DIFAUXCMD
%DIFDELCMD < \label{tbl:overall-experiments}
%DIFDELCMD <   \begin{tabular}{cccc}
%DIFDELCMD < %%%
\DIFdelendFL \DIFaddbeginFL \begin{description}
	\item[\DIFaddFL{Titanic Dataset}] \DIFaddFL{contains 100000 tourists and corresponding 12 attributes.
}\end{description}
\vspace{.3cm}
\begin{center}
	\DIFaddFL{\fbox{ 
		\parbox{1\textwidth}{ 
			\begin{center}
				\begin{itemize}
					\item
					First, we set up several classifier models to make a prediction. Then we use the
					training set to fit the model I built. After fitting, I use the fitted model to predict the remaining data in the training set and calculate its accuracy, weight, etc. Then fuse multiple groups of models, stack the fused model with the logistic regression model,and then fit the training set to get the prediction score. Use this model to predict our test set and see the prediction results of our test set.
				\end{itemize}
			\end{center}
		} 
	}
}\end{center}
\vspace{.5cm}
\begin{itemize}
	\item \DIFaddFL{we try Stacking Classifier and Logistic Regression on the test data.
}\end{itemize}
\vspace{.5cm}
\begin{center}
	\begin{tabular}{c| c c c c}
		\DIFaddendFL \toprule
		%DIF <  after \\: \hline or \cline{col1-col2} \cline{col3-col4} ...
    %DIF > \centering
		\DIFaddbeginFL {}  \DIFaddendFL & \DIFdelbeginFL \DIFdelFL{OR Event Detection }\DIFdelendFL \DIFaddbeginFL \texttt{\DIFaddFL{precision}} \DIFaddendFL & \DIFdelbeginFL \DIFdelFL{AC Event Detection }\DIFdelendFL \DIFaddbeginFL \texttt{\DIFaddFL{recall}}  \DIFaddendFL & \DIFdelbeginFL \DIFdelFL{TC Event Detection }\DIFdelendFL \DIFaddbeginFL \texttt{\DIFaddFL{f1-score}} & \texttt{\DIFaddFL{support}} \DIFaddendFL \\
		\midrule
		\DIFdelbeginFL \DIFdelFL{precision }\DIFdelendFL \DIFaddbeginFL \DIFaddFL{$0$
		}\DIFaddendFL &  \DIFdelbeginFL \DIFdelFL{0.83 }\DIFdelendFL \DIFaddbeginFL {\DIFaddFL{$0.79$}} \DIFaddendFL &  \DIFdelbeginFL \DIFdelFL{0.69 }\DIFdelendFL \DIFaddbeginFL {\DIFaddFL{$0.80$}} \DIFaddendFL &  \DIFdelbeginFL \DIFdelFL{0.46 }\DIFdelendFL \DIFaddbeginFL {\DIFaddFL{$0.79$}} &  {\DIFaddFL{$11336$}} \DIFaddendFL \\
		\DIFdelbeginFL \DIFdelFL{recall }\DIFdelendFL \DIFaddbeginFL \DIFaddFL{$1$
		}\DIFaddendFL &  \DIFdelbeginFL \DIFdelFL{0.68 }\DIFdelendFL \DIFaddbeginFL {\DIFaddFL{$0.73$}} \DIFaddendFL &  \DIFdelbeginFL \DIFdelFL{0.48 }\DIFdelendFL \DIFaddbeginFL {\DIFaddFL{$0.72$}} \DIFaddendFL &  \DIFdelbeginFL \DIFdelFL{0.36 }\DIFdelendFL \DIFaddbeginFL {\DIFaddFL{$0.72$}} &  {\DIFaddFL{$8664$}} \DIFaddendFL \\
		\DIFdelbeginFL \DIFdelFL{F-score }\DIFdelendFL \DIFaddbeginFL \DIFaddFL{accuracy
		}\DIFaddendFL &  \DIFdelbeginFL \DIFdelFL{0.747 }\DIFdelendFL \DIFaddbeginFL {} \DIFaddendFL &  \DIFdelbeginFL \DIFdelFL{0.57 }\DIFdelendFL \DIFaddbeginFL {} \DIFaddendFL &  \DIFdelbeginFL \DIFdelFL{0.4 }\DIFdelendFL \DIFaddbeginFL {\DIFaddFL{$0.76$}} &  {\DIFaddFL{$20000$}} \DIFaddendFL \\
		\DIFaddbeginFL \DIFaddFL{macro avg
		}&  {\DIFaddFL{$0.76$}} &  {\DIFaddFL{$0.76$}} &  {\DIFaddFL{$0.76$}} &  {\DIFaddFL{$20000$}} \\
		\DIFaddFL{weight avg
		}&  {\DIFaddFL{$0.76$}} &  {\DIFaddFL{$0.76$}} &  {\DIFaddFL{$0.76$}} &  {\DIFaddFL{$20000$}} \\
		\DIFaddendFL \bottomrule
	\end{tabular}
\DIFdelbeginFL %DIFDELCMD < \end{table}
%DIFDELCMD < %%%
\DIFdelend \DIFaddbegin \end{center}
\vspace{.5cm}
\begin{itemize}
	\item \DIFadd{First, let’s see if there are missing values in the test, and fill in the missing values in the test in the same way as train. Finally, the well-fitting model is used to make predictions.
}\end{itemize}
\DIFaddend 

\section{Conclusions} \label{sec-conclusions}
\DIFaddbegin \begin{itemize}
	\item \DIFadd{Finally, the highest accuracy in LogisticRegression Model is 0.76. }\\
	\DIFaddend 

	\DIFdelbegin %DIFDELCMD < \blindtext
%DIFDELCMD < %%%
\DIFdelend \DIFaddbegin \item \DIFadd{A relatively basic Kaggle project was selected, the purpose is to be familiar with the Kaggle project, deeply analyze and understand each line of the project process, this project has done more processing on the step of data feature processing, and learned a lot from it.
	}\DIFaddend 

	\DIFdelbegin \section*{\DIFdel{Acknowledgement}}
%DIFAUXCMD
%DIFDELCMD < 

%DIFDELCMD < \lipsum[1]
%DIFDELCMD < 

%DIFDELCMD < %%%
\DIFdel{The authors would like to thank \ldots
}%DIFDELCMD < 

%DIFDELCMD <  %%%
\DIFdelend \DIFaddbegin \item \DIFadd{The work can also be further refined to improve the accuracy of prediction, for example, in the process of processing the age column, the age can be segmented according to the size of the age, and it is felt that the size of the age has a certain relationship with the size of the final survival rate.
}\end{itemize}
 \DIFaddend